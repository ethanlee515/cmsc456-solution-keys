\documentclass{article}

\usepackage{amsmath, amssymb, amsthm}
\usepackage{braket, physics}
\usepackage{hyperref, cleveref}
\usepackage{algorithm}
\usepackage{xcolor}

\theoremstyle{definition}
\newtheorem{definition}{Definition}

\newcommand{\ga}[1]{{\noindent \textcolor{purple}{\emph{(GA: #1)}}}{}}

\title{Homework 3 Solution Key}
\author{Yi Lee}

\DeclareMathOperator{\XOR}{XOR}

\begin{document}

\newcommand{\sqp}[1]{\left[#1\right]}
\newcommand{\parens}[1]{\left(#1\right)}
\newcommand{\cD}{\mathcal{D}}
\newcommand{\distrof}[1]{\cD_{#1}}
\newtheorem{theorem}{Theorem}


\maketitle

\section*{Problem 1}

Suppose $\mathsf{Enc}$ takes $n$-bit messages.
Consider the following adversary against the IND-CPA game:
\begin{enumerate}
    \item Send $m_0=0^n$ and $m_1=0^{n-1}1$ to the challenger, and receive $c$.
    \item Send $m_0$ to the encryption oracle, and receive $c_0$.
    \item If $c=c_0$, output $0$. Otherwise, output $1$.
\end{enumerate}
By construction, this adversary wins the IND-CPA game with certainty.

\section*{Problem 2}

Consider the the same adversary $\mathcal{A}$ as in Problem 1 against the IND-CPA game:
\begin{enumerate}
    \item Send $m_0=0^n$ and $m_1=0^{n-1}1$ to the challenger, and receive $c$.
    \item Send $m_0$ to the encryption oracle, and receive $c_0$.
    \item If $c=c_0$, output $b'=0$. Otherwise, output $b'=1$.
\end{enumerate}
We claim that this adversary wins the IND-CPA game with non-negligible advantage.

Suppose that the challenger samples some randomness $r$ during encryption, and the encryption oracle samples some $r'$ during encryption.
Note that there are $2^{\log n} = n$ possible $r$'s.
Moreover, $r$ and $r'$ are sampled independently and uniformly randomly.
We therefore have $\Pr\sqp{r=r'}=\frac{1}{n}$.
The winning probability of the adversary is then
\begin{align*}
    \Pr\sqp{CPA[\Pi, \mathcal{A}] = 1} &= \Pr\sqp{b = b'} \\
    &=\Pr\sqp{b = b' = 0} + \Pr\sqp{b = b' = 1} \\
    &=\frac{1}{2} \Pr\sqp{b' = 0 | b = 0} + \frac{1}{2}\Pr\sqp{b' = 1 | b = 1} \\
    &=\frac{1}{2} \parens{\Pr\sqp{b'=0 \wedge r=r' | b=0} + \Pr\sqp{b'=0 \wedge r\ne r' | b=0}} \\
    &\qquad + \frac{1}{2}\Pr\sqp{b' = 1 | b = 1}\\
    &= \frac{1}{2} \Pr\sqp{b'=0 | r=r' \wedge b=0}\cdot \Pr\sqp{r=r' | b=0} \\
    &\qquad + \frac{1}{2} \Pr\sqp{b'=0 | r\ne r' \wedge b=0}\cdot \Pr\sqp{r\ne r' | b=0} \\
    &\qquad + \frac{1}{2}\Pr\sqp{b' = 1 | b = 1} \\
    &=\frac{1}{2}\cdot1\cdot \frac{1}{n} + \frac{1}{2}\cdot 0\cdot \parens{1-\frac{1}{n}} + \frac{1}{2} \\
    &= \frac{1}{2} + \frac{1}{2n}.
\end{align*}
We conclude that our attacker wins with non-negligible advantage, and therefore the scheme is not IND-CPA secure.

\section*{Problem 3}

Consider the following adversary.
\begin{enumerate}
    \item Send the messages $m_0=0^n$ and $m_1=0^{n-1}1$ to the challenger, and receive $c$.
    \item Parse $(r, a)\gets c$.
    \item Send $(m_0, r)$ to $\mathcal{O}_k$, receive $c_0$.
    \item If $c=c_0$, output $0$. Otherwise, output $1$.
\end{enumerate}
By construction, this adversary breaks IND with certainty.
More concretely, \begin{itemize}
    \item If $b=0$, the adversary outputs $b'=0$ by construction.
    \item If $b=1$, then $c_0$ and $c$ are valid ciphertexts corresponding to different plaintexts.
    By correctness then, we must have $c_0\ne c$ and therefore $b'=1$.
\end{itemize}

\section*{Problem 4}

\subsection*{Part a}

No.
Suppose the adversary queries the decryption oracle with some chosen input $c=(r^*, a)$, and receives some message $m$ as the decryption result.
In that case, the adversary learns no more than $F_k(r^*)=a\oplus m$.
In other words, for all $r\ne r^*$, $F_k(r)$ still appears uniformly random in the view of the adversary.

Throughout the rest of the interaction,
the challenger and the oracle encrypt messages chosen by the adversary.
With overwhelming (that is, all but negligible) probability,
they sample $r_i$ during encryption that satisfy $r_i\ne r^*$ for all $r_i$.
All the ciphertexts received by the adversary therefore are one-time padded by $F_k(r_i)$ and contain no useful information.

We conclude that the adversary learns nothing about $b$, and cannot guess it with non-negligible advantage.

\subsection*{Part b}

Yes.
Consider the following attacker.
\begin{enumerate}
    \item Send the messages $m_0=0^n$ and $m_1=0^{n-1}1$ to the challenger, and receive $c$.
    \item Send $c$ to the decryption oracle, receive $m$.
    \item If $m=m_0$, output $0$. Otherwise, output $1$.
\end{enumerate}
By construction, this attacker wins with certainty.

\subsection*{Part c}

Yes.
Consider the following attacker.
\begin{enumerate}
    \item Send the messages $m_0=0^n$ and $m_1=0^{n-1}1$ to the challenger, and receive $c$.
    \item Parse $(r, a)\gets c$.
    \item Evaluate $a'\gets a\oplus 0^{n-1}1$.
    \item Send $(r, a')$ to the decryption oracle, receive $m$.
    \item If $m=0^{n-1}1$, output $1$. Otherwise, output $0$.
\end{enumerate}
We claim that this attacker wins with certainty.
\begin{itemize}
    \item If $b=0$, we have $a=F_k(0^n)$.
    We then have
    \begin{align*}
        \mathsf{Dec}_k((r, a'))&=a' \oplus F_k(r) \\
        &= F_k(r)\oplus 0^{n-1}1\oplus F_k(r) \\
        &= 0^{n-1}1
    \end{align*}
    and $b'=0$ with certainty.
    \item If $b=1$, we have $a=F_k(r)\oplus 0^{n-1}1$ and $a'=F_k(r)$.
    We then have
    \begin{align*}
        \mathsf{Dec}_k((r, a'))&=a' \oplus F_k(r) \\
        &= F_k(r)\oplus F_k(r) \\
        &= 0^n
    \end{align*}
    and $b'=1$ with certainty.
\end{itemize}

\end{document}