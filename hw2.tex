\documentclass{article}

\usepackage{amsmath, amssymb, amsthm}
\usepackage{braket, physics}
\usepackage{hyperref, cleveref}
\usepackage{algorithm}
\usepackage{xcolor}

\theoremstyle{definition}
\newtheorem{definition}{Definition}

\newcommand{\ga}[1]{{\noindent \textcolor{purple}{\emph{(GA: #1)}}}{}}

\title{Homework 2 Solution Key}
\author{Yi Lee}

\DeclareMathOperator{\XOR}{XOR}

\begin{document}

\newcommand{\sqp}[1]{\left[#1\right]}
\newcommand{\parens}[1]{\left(#1\right)}
\newcommand{\cD}{\mathcal{D}}
\newcommand{\distrof}[1]{\cD_{#1}}
\newtheorem{theorem}{Theorem}


\maketitle

\section*{Problem 1}

\subsection*{Part a}

No, $G$ is not a pseudorandom generator.
Consider the following distinguisher $D(x)$:
\begin{enumerate}
    \item Unpack the input as $x=x_0||x_1$
    \item Compute $r^* = x_0 \oplus x_1$
    \item If $r^* = 0101\cdots01$, output $1$.
    Otherwise output $0$.
\end{enumerate}

We then have
$$\Pr_{s\gets\set{0, 1}^n}q\sqp{D(G(s))=1}=1$$
by construction.
On the other hand, we have
$$\Pr_{y\gets\set{0, 1}^{2n}}\sqp{D(y)=1}=\frac{1}{2^{n}},$$
as $r^*$ is uniformly random and compared with a fixed length-$n$ string $r=0101\cdots01$.

\subsection*{Part b}

No, $G$ is not a pseudorandom generator.
Consider the following distinguisher $D(x)$:
\begin{enumerate}
    \item Unpack the input as $x=x_0||x_1$
    \item If $f(x_0)=x_1$, output $1$.
    Otherwise output $0$.
\end{enumerate}
We then have
$$\Pr_{s\gets\set{0, 1}^n}q\sqp{D(G(s))=1}=1$$
by construction.
On the other hand, we claim
$$\Pr_{r\gets\set{0, 1}^{2n}}\sqp{D(r)=1}=\frac{1}{2^{n}}.$$
Observe that $x_1$ is uniformly random and $f(x_0)$ is independent from it.
The probability they're equal is therefore simply the number of possible values of $x_1$.

\subsection*{Part c}

$G$ is a pseudorandom generator.
We will prove this by reduction:
suppose for the sake of contradiction that there exists a distinguisher $\mathcal{D}$ against $G$, and we construct a distinguisher $\mathcal{D}'$ against $H$.
Our distinguisher $\mathcal{D}'(x)$ simply sends its input $x$ to $\mathcal{D}$ and output whatever it $\mathcal{D}(x)$ outputs.

We then have the following bound:
\begin{align*}
   \Pr\sqp{\mathcal{D}(G(s))=1} &= \Pr\sqp{\mathcal{D}(G(s))=1\wedge s=0^n}+\Pr\sqp{\mathcal{D}(G(s))=1\wedge s\ne0^n} \\
   &= \frac{1}{2^n} \Pr\sqp{\mathcal{D}'(0^{2n})=1} + \Pr\sqp{\mathcal{D}'(H(s))=1\wedge s\ne 0^n} \\
   &= \frac{1}{2^n} \Pr\sqp{\mathcal{D}'(0^{2n})=1} + \Pr\sqp{\mathcal{D}'(H(s))=1}
   \\
   &\qquad - \Pr\sqp{\mathcal{D}'(H(s))=1 \wedge s=0^n} \\
   &= \frac{1}{2^n} \Pr\sqp{\mathcal{D}'(0^{2n})=1} + \Pr\sqp{\mathcal{D}'(H(s))=1} \\
   &\qquad - \frac{1}{2^n}\Pr\sqp{\mathcal{D}'(H(0^n))=1}.
\end{align*}
We therefore have
\begin{equation*}
    \abs{\Pr\sqp{\mathcal{D}(G(s))=1} - \Pr\sqp{\mathcal{D}'(H(s))=1}} < \frac{1}{2^n}.
\end{equation*}
By triangle inequality then, we can lower bound the advantage of our distinguisher $\mathcal{D}'$
\begin{align*}
    \abs{\Pr\sqp{\mathcal{D}'(H(s))=1} - \Pr\sqp{\mathcal{D}'(r)=1}} \geq \abs{\Pr\sqp{\mathcal{D}(G(s)=1} - \Pr\sqp{\mathcal{D}(r)=1}} - \frac{1}{2^n}
\end{align*}
which is not negligible.

\section*{Problem 2}

We have the following definition.
\begin{definition}
    \label{def:prg}
    A function $G$ is a PRG if for all efficient adversaries $\mathcal{A}$,
$$\abs{\Pr\sqp{\mathsf{PRGUESS}\sqp{G, \mathcal{A}} = 1}-\frac{1}{2}}\leq \mathsf{negl}(n).$$
\end{definition}

We now show that \Cref{def:prg} is equivalent to the textbook definition.
First observe that
\begin{align*}
    \Pr\sqp{\mathsf{PRGUESS}\sqp{G, \mathcal{A}} = 1} &= \Pr\sqp{\mathsf{PRGUESS}\sqp{G, \mathcal{A}} = 1 \wedge b=0} \\
    &\qquad + \Pr\sqp{\mathsf{PRGUESS}\sqp{G, \mathcal{A}} = 1 \wedge b=1} \\
    &= \frac{1}{2} \Pr\sqp{\mathsf{PRGUESS}\sqp{G, \mathcal{A}} = 1 | b=0} \\
    &\qquad + \frac{1}{2} \Pr\sqp{\mathsf{PRGUESS}\sqp{G, \mathcal{A}} = 1 | b=1} \\
    &= \frac{1}{2}\Pr\sqp{\mathcal{A}(r)=0} + \frac{1}{2}\Pr\sqp{\mathcal{A}(G(s))=1} \\
    &=\frac{1}{2}\Pr\sqp{\mathcal{A}(G(s))=1} +\frac{1}{2} - \frac{1}{2} \Pr\sqp{\mathcal{A}(r)=1}.
\end{align*}
We therefore have
$$\abs{\Pr\sqp{\mathsf{PRGUESS}\sqp{G, \mathcal{A}} = 1}-\frac{1}{2}} = \frac{1}{2}\abs{\Pr\sqp{\mathcal{A}(G(s))=1} - \Pr\sqp{\mathcal{A}(r)=1}}.$$
Notice that the expression above is syntactically similar to the textbook definition.
The only difference is the extra multiplicative factor of $\frac{1}{2}$, which makes no difference as to whether a function is negligible.

\section*{Problem 3}

The scheme is \emph{not} IND.
Consider the following adversary:
\begin{enumerate}
    \item Choose the messages $m_0=0^{n+1}$ and $m_1=0^n1$ for the challenger.
    \item On receiving the ciphertext $c$, unpack it as $c=k^*||x$ where $k^*\in\set{0, 1}^n$ and $x\in\set{0, 1}$.
    Compute and output $b'=x\oplus \XOR(k^*)$.
\end{enumerate}
By construction, this adversary wins the IND experiment with probability $1$.
More concretely, it computes $k^*=k$ correctly,
then we have
\begin{align*}
   b' &= c_{n+1}\oplus\XOR(k) \\
   &= b \oplus \XOR(k)\oplus\XOR(k) \\
   &= b
\end{align*}
with certainty.

\section*{Problem 4}

\subsection*{Part a}

$F$ is \emph{not} a PRF.
Consider the following adversary:
\begin{enumerate}
    \item Send the input $z=0^n$ to its oracle and receive some output $y\in\set{0, 1}^{2n}$
    \item Let $y_0$ be the first $n$ of $y$.
    If $y_0=0^n$, then output $1$;
    otherwise, output $0$.
\end{enumerate}
We then have
$\Pr\sqp{A^{F_k}(1^n)=1}=1$
by construction.
On the other hand, we have
$\Pr\sqp{A^{R}(1^n)=1}=\frac{1}{2^n}$,
as $y_0$ would be a uniformly random $n$-bit string.

\subsection*{Part b}

$F$ is \emph{not} a PRF, as it is not a deterministic function.

Alternatively, one may want to implement $F$ by sampling its entire input/output table up-front.
In that case, $F$ would be inefficient, as there are exponentially many input-output pairs.

\subsection*{Part c}

We will show that with certain choices of $G$, $F$ is not a PRF.
More concretely, for all PRG $G$, there exists another PRG $G'$ so that $F_k(z)=G'(k || z)$ is not a PRF.
The PRG $G'(x)$ can be defined as follows:
\begin{enumerate}
    \item Unpack $x = k || z$
    \item If $z = 0^n$, output $0^\ell$; otherwise, output $G(x)$.
\end{enumerate}

This can be shown to be a PRG by the same argument as in part (1)(c).
On the other hand, to show that $F_k(z)=G'(k || z)$ is not a PRF, consider the following adversary: \begin{enumerate}
    \item Send $x=0^n$ to the oracle, and receive $y_0$.
    \item Output $1$ if $y_0=0^\ell$, otherwise output $0$.
\end{enumerate}
We clearly have $\Pr\sqp{A^{F_k}(1^n)=1}=1$ and $\Pr\sqp{A^R(1^n)=1}=\frac{1}{2^\ell}$ for this adversary.

\subsection*{Part d}

$F$ is \emph{not} a PRF.
Consider the following adversary:
\begin{enumerate}
    \item Send the input $x=0^n$ to its oracle and receive some output $y_0\in\set{0, 1}^n$
    \item Send the input $x=0^{n-1}1$ to its oracle and receive some output $y_1\in\set{0, 1}^n$.
    \item Compute $H_{y_0}(0^{n-1}1)$ locally.
    If $H_{y_0}(0^{n-1}1)=y_1$, output $1$;
    otherwise, output $0$.
\end{enumerate}
We then have
$\Pr\sqp{A^{F_k}(1^n)=1}=1$
by construction.
On the other hand, we also have
$\Pr\sqp{A^{R}(1^n)=1}=\frac{1}{2^n}$ as $y_1$ would be uniformly random and independent from $H_{y_0}(0^{n-1}1)$ in this case.

\section*{Problem 5}

Suppose there exists a nontrivial pseudorandom function $F_k$ with key length $n_k$, input length $n_i$, and output length $n_\ell$.
We then have
$n_k < 2^{n_i} n_\ell$;
otherwise a truly random function can be constructed trivially by writing the key $k$ onto the output table.

We now define the number of calls $a = \lceil\frac{n_k}{n_\ell}\rceil$ and the function $$G(x)= F_x(0)||F_x(1)||\ldots ||F_x(a - 1)$$
where the inputs to $F$ are encoded in binary.
Notice that this is well-defined, as we indeed have $\frac{n_k}{n_\ell}\leq 2^{n_i}$;
moreover, this makes the output of $G(x)$ longer than its input.
We will now prove $G(x)$ is a PRG by reduction:
suppose that there exists an adversary $\mathcal{A}$ against $G$, then we will construct $\mathcal{B}$ against $F_k$ as follows:
\begin{enumerate}
    \item For each $0\leq i < a$, send the input $i$ to its oracle and receive some result $y_i$.
    \item Set $y=y_0 || y_1 || \ldots || y_{a-1}$ as the concatenation of all $y_i$'s.
    \item Run $\mathcal{A}(y)$ and output whatever it outputs.
\end{enumerate}
We then have
\begin{align*}
   \Pr\sqp{\mathcal{B}^{F_k}(1^n)=1}&=\Pr\sqp{\mathcal{A}(y)=1} \\
   &=\Pr[\mathcal{A}(G(k))=1]
\end{align*}
and
\begin{align*}
    \Pr\sqp{\mathcal{B}^R(1^n)=1} &= \Pr\sqp{\mathcal{A}(r)=1}.
\end{align*}
We therefore have
$$\abs{\Pr\sqp{\mathcal{B}^{F_k}(1^n)=1} - \Pr\sqp{\mathcal{B}^R(1^n)=1}} =
\abs{\Pr\sqp{\mathcal{A}(G(k))=1} - \Pr\sqp{\mathcal{A}(r)=1}}.$$

\end{document}