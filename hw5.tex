\documentclass{article}

\usepackage{amsmath, amssymb, amsthm}
\usepackage{braket, physics}
\usepackage{hyperref, cleveref}
\usepackage{algorithm}
\usepackage{xcolor}

\theoremstyle{definition}
\newtheorem{definition}{Definition}

\newcommand{\ga}[1]{{\noindent \textcolor{purple}{\emph{(GA: #1)}}}{}}

\title{Homework 5 Solution Key}
\author{Yi Lee}

\DeclareMathOperator{\Sym}{Sym}

\begin{document}

\newcommand{\sqp}[1]{\left[#1\right]}
\newcommand{\parens}[1]{\left(#1\right)}
\newcommand{\cD}{\mathcal{D}}
\newcommand{\distrof}[1]{\cD_{#1}}
\newtheorem{theorem}{Theorem}


\maketitle

\section*{Problem 1}

TODO

\section*{Problem 2}

TODO

\section*{Problem 3}

\subsection*{Part a}

We show that all ciphertexts are equally likely.
For all messages $m$ and ciphertext $c$, we have
\begin{align*}
    \Pr_{k}[c=\mathsf{Enc}(m)]&=\Pr_{P\in\Sym(2^n)}[P(m)=c]\\
    &=\frac{1}{2^n}
\end{align*}
since there are $2^n$ equally likely outcomes that $P(m)$ could map to.

\subsection*{Part b}

Consider the ``lazy sampling" hybrid as follows:
\begin{itemize}
    \item The oracle keeps a function table $f\subseteq \set{0, 1}^n\times \set{0, 1}^n$
\end{itemize}

\subsection*{Part c}

TODO

\end{document}