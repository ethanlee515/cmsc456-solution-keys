\documentclass{article}

\usepackage{amsmath, amssymb, amsthm}
\usepackage{braket, physics}
\usepackage{hyperref, cleveref}
\usepackage{algorithm}
\usepackage{xcolor}

\theoremstyle{definition}
\newtheorem{definition}{Definition}

\newcommand{\ga}[1]{{\noindent \textcolor{purple}{\emph{(GA: #1)}}}{}}

\title{Midterm Solution Key}
\author{Yi Lee}

\DeclareMathOperator{\XOR}{XOR}

\begin{document}

\newcommand{\sqp}[1]{\left[#1\right]}
\newcommand{\parens}[1]{\left(#1\right)}
\newcommand{\cD}{\mathcal{D}}
\newcommand{\distrof}[1]{\cD_{#1}}
\newtheorem{theorem}{Theorem}


\maketitle

\section*{Problem 1}

No, the result is \emph{not} perfectly secret.
More concretely, we show that not all ciphertexts are equally likely.

Consider $m_0=0^n$ and $m_1=0^{n-1}1$.
We have
$\Pr\sqp{\mathsf{Enc}(m_0)=0^n}=0$
and
$\Pr\sqp{\mathsf{Enc}(m_1)=0^n}=\frac{1}{2^n}$.

\section*{Problem 2}

Let $\mathcal{D}$ be any arbitrary distinguisher.

We have
\begin{align*}
    \abs{\Pr_s\sqp{\cD(L(s))=1}-\Pr_r\sqp{\cD(r)=1}} &= \abs{\Pr_s\sqp{\cD(H(G(s)))=1}-\Pr_r\sqp{\cD(r)=1}} \\
    &\leq \abs{\Pr_s\sqp{\cD(H(G(s)))=1}-\Pr_{r'}\sqp{\cD(H(r'))=1}} \\
    &\qquad + \abs{\Pr_{r'}\sqp{\cD(H(r')=1}-\Pr_r\sqp{\cD(r)=1}} \\
    &= \abs{\Pr_s\sqp{(\cD\circ H)(G(s))=1}-\Pr_{r'}\sqp{(\cD\circ H)(r')=1}} \\
    &\qquad + \abs{\Pr_{r'}\sqp{\cD(H(r')=1}-\Pr_r\sqp{\cD(r)=1}}
\end{align*}
where we first apply triangle inequality,
then rewrite $\cD'=(\cD\circ H)$ as a composition of two programs.
We conclude that the right hand side is negligible, as it is the sum of two negligible functions.

\section*{Problem 3}

\subsection*{Part a}

Consider the following adversary.
\begin{enumerate}
    \item Send $x=0^n$ to the oracle, receive $y_1$.
    \item Send $x=1^n$ to the oracle, receive $y_2$.
    \item Unpack $(a_1, b_1)\gets y_1$ and $(a_2, b_2)\gets y_2$.
    \item If $a_1=b_2$ and $b_1=a_2$, output $0$. Otherwise, output $1$.
\end{enumerate}
This adversary wins with all but negligible probability:
\begin{itemize}
    \item On the oracle being $H$, it wins with certainty by construction.
    \item On the oracle being truly random, $a_1, a_2, b_1, b_2$ are all uniformly random and independent.
    The adversary therefore answers $0$ with probability $\frac{1}{2^{2n}}$.
\end{itemize}

\subsection*{Part b}

Consider the following adversary.
\begin{enumerate}
    \item Send $x=0^{n-1}$ to the oracle, receive $y_1$.
    \item Send $x=0^{n-2}1$ to the oracle, receive $y_2$.
    \item Unpack $(a_1, b_1)\gets y_1$ and $(a_2, b_2)\gets y_2$.
    \item If $b_1=a_2$, output $0$. Otherwise, output $1$.
\end{enumerate}
This adversary wins with all but negligible probability.
\begin{itemize}
    \item On the oracle being $H$, it wins with certainty by construction.
    \item On the oracle being truly random, $b_1$ and $a_2$ are uniformly random and independent.
    The adversary therefore answers $0$ with probability $\frac{1}{2^{n}}$.
\end{itemize}

\section*{Problem 4}

We will restate the scheme from the lecture and show its correctness.

Let $F:\set{0, 1}^n\times \set{0, 1}^\ell \rightarrow \set{0, 1}^t$ be a PRF.
\begin{itemize}
    \item Keygen: On input $1^n$, sample and output a uniformly random PRF key $k\gets\set{0, 1}^n$.
    \item Encryption. On inputs $k$ and $m\in\set{0, 1}^t$: \begin{enumerate}
        \item Sample a random string $r\gets\set{0, 1}^\ell$
        \item Evaluate $a\gets F_k(r) \oplus m$
        \item Output $c\gets (r, a)$
    \end{enumerate}
    \item Decryption. On inputs $k$ and $c$: \begin{enumerate}
        \item Unpack $(r, a) \gets c$
        \item Output $m\gets a\oplus F_k(r)$
    \end{enumerate}
\end{itemize}

We now show correctness.
For all key $k$ and message $m$, we have
\begin{align*}
    \mathsf{Dec}_k(\mathsf{Enc}_k(m)) &= \mathsf{Dec}_k(r, F_k(r)\oplus m) \\
    &= F_k(r)\oplus m \oplus F_k(r) \\
    &= m
\end{align*}

\section*{Problem 5}

We restate the MAC construction from the lecture.
Let $F:\set{0, 1}^n\times \set{0, 1}^\ell \rightarrow \set{0, 1}^t$ be a PRF.
\begin{itemize}
    \item Keygen: On input $1^n$, sample and output a uniformly random PRF key $k\gets\set{0, 1}^n$.
    \item Tag generation: On inputs $k\in\set{0, 1}^n$ and $m\in\set{0, 1}^\ell$, output $F_k(m)$.
\end{itemize}

\end{document}