\documentclass{article}

\usepackage{amsmath, amssymb, amsthm}
\usepackage{braket}
\usepackage{hyperref, cleveref}
\usepackage{algorithm}
\usepackage{xcolor}

\newcommand{\ga}[1]{{\noindent \textcolor{purple}{\emph{(GA: #1)}}}{}}

\title{Homework 1 Solution Key}
\author{Yi Lee}

\DeclareMathOperator{\Bern}{Bern}

\begin{document}

\newcommand{\sqp}[1]{\left[#1\right]}
\newcommand{\parens}[1]{\left(#1\right)}
\newcommand{\cD}{\mathcal{D}}
\newcommand{\distrof}[1]{\cD_{#1}}
\newtheorem{theorem}{Theorem}


\maketitle

\section*{Problem 1}

\noindent \emph{Solution.} Suppose we have a ciphertext from a broken historical cipher. The key observation is that the total number of historical ciphers is quite small, so one can simply try all the known attacks. 
That is, given a ciphertext $c$, we perform:
\begin{enumerate}
    \item let $m$ be the result of the brute-force attack on the shift cipher, with input $c$. If $m$ is a valid sentence in English, terminate and output $m$.
    \item let $m$ be the result of the frequency analysis attack for the Vigen\`ere cipher, with input $c$. If $m$ is a valid sentence in English, terminate and output $m$.
    \item etc.
\end{enumerate}

~\\ \noindent \emph{Discussion.} In modern times, there are many more broken schemes out there, but the total number of them is still quite small relative to the power of modern computers. So a ``try-all-known-attacks'' algorithm is still feasible. Typically a clever cryptanalyst can even break a scheme that has never been published before; some tools (like frequency analysis) can reveal a lot of structure even if nothing is known about the scheme itself.


\section*{Problem 2}

\subsection*{Part a}

We describe the scytale encryption scheme in \Cref{alg:scytale}.

\begin{algorithm}
    \caption{The Scytale Encryption}
    \label{alg:scytale}
    Key generation: On input $1^n$, sample and output a ``rod circumference" $\ell\gets\set{1, 2, \ldots, n - 1}$ randomly.\\
    
    Encrypt: For simplicity, we assume the length of the message is some multiple of $\ell$, and we write $n=k\ell$. This can always be achieved by padding the message with $0$s if necessary.
    On inputs key $\ell$ and (zero-indexed) message $m\in\set{0, 1}^{k\ell}$,
    \begin{enumerate}
        \item Compute the ``rod content" matrix $R\in\set{0, 1}^{k\times\ell}$:
        for each $0\leq x < k$ and $0\leq y < \ell$, set $R_{xy} = m_{x+\ell y}$.
        \item For $0\leq i< k\ell$, compute $x_i=\lfloor\frac{i}{\ell}\rfloor$ and $y_i=\parens{i\mod \ell}$, and set $c_i=R_{x_iy_i}$.
        \item Output the string $c=c_0c_1\ldots c_{k\ell-1}$.
    \end{enumerate}
    Decrypt: On inputs $\ell$ and $c\in\set{0, 1}^{k\ell}$,
    \begin{enumerate}
        \item Compute the ``rod content" matrix $R\in\set{0, 1}^{k\times\ell}$:
        for each $0\leq x < k$ and $0\leq y < \ell$, set $R_{xy} = c_{y+\ell x}$.
        \item For $0\leq i < n$, compute $x_i=\parens{i\mod \ell}$ and $y_i=\lfloor\frac{i}{\ell}\rfloor$, and set $m_i=R_{x_iy_i}$.
        \item Output the string $m=m_0m_1\ldots m_{n-1}$.
    \end{enumerate}
\end{algorithm}

\subsection*{Part b}

No. Some ways to see this:
\begin{itemize}
    \item The encryptions of $0^n$ and $1^n$ are trivially distinguishable.
    \item Encryption preserves the ``Hamming weight" (i.e., the number of $1$s) of the message. This implies that not all ciphertexts are equally likely.
    \item One can break the scheme simply by trying all possible rod circumferences until the outcome message makes sense in English.
\end{itemize}



\section*{Problem 3}

False. Suppose an adversary had some ``brute force'' method that, starting with a ciphertext $c$, produces a guess $k'$ for the key which is better than random guessing. In the OTP, the ciphertext and the key together fully determine the plaintext. Thus it follows that, by also outputting $m' := k' \oplus c$, the adversary could \emph{also} produce a guess $m'$ for the plaintext which is better than random guessing. This contradicts perfect secrecy which states that random guessing is the best strategy even after seeing the ciphertext $c$ (i.e., $\Pr[M=m | C=c] = \Pr[M=m]$). Note that this argument does not assume any particular distribution on messages.


%Note that each ciphertext $c$ 
%Given a ciphertext $c$, suppose an attacker checks all eight possible keys $(k_1, k_2, \ldots, k_8)$.
%Recall that $c\oplus k = m$;
%that is, for all fixed ciphertext $c$, there is a one-to-one correspondance between the key $k$ and the message $m$.
%The attack is therefore equivalent to checking all possible messages $(m_1, m_2, \ldots, m_8)$ by brute force.
%Recall however that we have
%$$\Pr[M=m | C=c] = \Pr[M=m]$$
%for all fixed $m, c$,
%which shows that the ciphertext $c$ does not help in this setting.


\section*{Problem 4}

\subsection*{Part a}

Let $\mathcal{D}_M$ be the distribution of messages.
% Let $f^{-1}(s) = \set{x\in\set{0, 1}^n : f(x)=s}\subseteq\set{0, 1}^n$ denote the preimage set of $s$ under $f$.
We then have
\begin{align*}
   \Pr\sqp{f(M)=s | C=c}
   &=\Pr_{\substack{M\gets\mathcal{D}_M \\ k\gets\set{0, 1}^n}}\sqp{f(M)=s | M\oplus k=c} \\
    &=\Pr_{\substack{M\gets\mathcal{D}_M \\ k\gets\set{0, 1}^n}}\sqp{f(M)=s | k=M\oplus c} \\
    &= \frac{\Pr\sqp{f(M)=s \wedge k=M\oplus c}}{\Pr\sqp{k=M\oplus c}}
\end{align*}
We interpret the numerator operationally as first sampling $M$ and checking $f(M) = s$, then sampling $k$ and checking if it is precisely $M\oplus c$.
We therefore have
\begin{equation*}
    \Pr_{\substack{M\gets\mathcal{D}_M \\ k\gets\set{0, 1}^n}}\sqp{f(M)=s \wedge k=M\oplus c} = \Pr_{\substack{M\gets\mathcal{D}_M}}\sqp{f(M)=s}\cdot \frac{1}{2^n}.
\end{equation*}
We interpret the denominator as first sampling $M$, then sampling $k$ and checking if it is precisely $M\oplus c$.
We have
\begin{equation*}
    \Pr_{\substack{M\gets\mathcal{D}_M \\ k\gets\set{0, 1}^n}}\sqp{k=M\oplus c} = \frac{1}{2^n}.
\end{equation*}
Putting everything together, we conclude that
\begin{equation}
    \label{eq:otp-property}
    \Pr\sqp{f(M)=s | C=c} =
    \Pr\sqp{f(M)=s}.
\end{equation}

\subsection*{Part b}

We interpret the expression $f(M)=s$ as that the message $M$ satisfies some property $s$ that can be computed using $f$.
This intuitively captures anything the adversary is interested in learning about the message.
For example, maybe $M$ is some long sequence of internal government messages, and $f(M)$ is the result of applying some search pattern or LLM algorithm (say, how many times ``Eastasia'' appears in $M$.).

As unknown (or partial) information is modeled using distributions, $f(\distrof{M})$ then corresponds to a guess on whatever properties satisfied by an unseen message.
\Cref{eq:otp-property} shows that the distribution $f(\distrof{M})$ is independent from the ciphertext $C$.
In other words, by receiving $C=c$, the adversary gains no additional information beyond what it can compute locally from $f(\distrof{M})$ for a random message.

\section*{Problem 5}

\subsection*{Part a}

Let $f(n)$ and $g(n)$ be polynomials.
By definition, there exists $c_f$ so that $f(n) = O(n^{c_f})$,
and similarly for $c_g$.
We then have
\begin{align*}
    f(n)+g(n) &= O(n^{c_f} + n^{c_g}) \\ &=O(n^{\max(c_f, c_g)})
\end{align*} and $f(n)g(n) = O(n^{c_f+c_g})$.

\subsection*{Part b}

Let $p(n)$ be a polynomial.
We will find an $N$ such that $f(n)+g(n) < \frac{1}{p(n)}$ for all $n>N$.
As $f(n)$ is negligible, and $2p(n)$ is polynomial as well, there exists $N_f$ so that for all $n > N_f$ we have $f(n)<\frac{1}{2p(n)}$.
Define $N_g$ similarly.
We then have for all $n > N := \max(N_f, N_g)$, $f(n)+g(n)<\frac{1}{p(n)}$.

\subsection*{Part c}

Let $q(n)$ be a polynomial.
As $f(n)$ is negligible and $p(n)q(n)$ is a polynomial, there exists some $N$ so that $f(n) < \frac{1}{p(n) q(n)}$ for all $n > N$.
It follows that $f(n)p(n)<\frac{1}{q(n)}$.

\section*{Problem 6}

\subsection*{Part a}

Observe that the event that \emph{at least one} trial succeeds is equivalent to that \emph{any} trial succeeds, so we can write
\begin{equation*}
    E_n' = \bigvee_{i=1}^n E_i,
\end{equation*}
as ``(trial 1 succeeds) or (trial 2 succeeds) or ...".
We therefore have
\begin{align}
    \Pr\sqp{E_n'}&=\Pr\sqp{\bigvee_{i=1}^n E_i} \\
    &\leq \sum_{i=1}^n\Pr\sqp{E_i} \label{eq:union-bound} \\
    & \leq n \cdot \max_{i\in\set{1,\ldots, i}} \Pr\sqp{E_i}
\end{align}
where \Cref{eq:union-bound} is due to the union bound.
As each $\Pr\sqp{E_i}$ is negligible, so is their maximum.
Finally, we have shown in Problem 5(c) that a polynomial multiplied by a negligible function is negligible.

\subsection*{Part b}

Suppose an adversary tries an attack repeatedly.
Let $E_i$ be the event of the $i$-th attack succeeds.
By the analysis above, the probability of $E_n'$ corresponds to at least one attempt succeeds, which is negligible.

\section*{Problem 7}

\subsection*{Part a}

We recall the ``Chernoff bound" from the textbook Proposition A.14:
\begin{theorem}[Chernoff bound, rephrased]
    \label{thm:chernoff}
    Fix $\varepsilon>0$, $b\in\set{0, 1}$, and $m\in\mathbb{N}$.
    For $1\leq i\leq m$, suppose $X_i=\Bern(\frac{1}{2}+\varepsilon)$ are all independent.
    The probability of their majority value is not $b$ is then at most $e^{-\varepsilon^2m/2}$.
\end{theorem}

Now, choose $m=\lceil 2n\ln 2\cdot q^2(n) \rceil$.
We define $\mathcal{B}$ to simply run $\mathcal{A}$ for $m$ times on the same input $x$, and output their majority value.
(That is, if at least $\frac{m}{2}$ repetitions of $\mathcal{A}$ outputs $1$, then $\mathcal{B}$ outputs $1$;
otherwise $\mathcal{B}$ outputs $0$.)
By \Cref{thm:chernoff} then, we have
$$\Pr\sqp{\mathcal{B}(x)\ne b}\leq e^{-\varepsilon^2m/2}\leq 2^{-n}.$$

\subsection*{Part b}

We have shown that if an adversary can attack repeatedly, then it can potentially break a scheme with overwhelming probability.
For example, suppose that an adversary has a noticeable advantage in distinguishing $\mathsf{Enc}(0)$ from $\mathsf{Enc}(1)$,
then the reduction above may be useful for distinguishing the ensemble of encryptions $\set{\mathsf{Enc}(0)}_{i=1}^n$ from $\set{\mathsf{Enc}(1)}_{i=1}^n$ with much better advantage.

\end{document}